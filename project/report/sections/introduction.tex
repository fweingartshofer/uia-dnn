\section{Introduction}\label{sec:introduction}

Classifying human activities is a task performed daily by various devices such as smartwatches and smartphones.
These devices offer a lot of sensor data that can be used in machine learning contexts.
With this project, we focus on using accelerometer data to classify human activities.
We selected this project because it aligns well with the degree we are currently pursuing at our home university called \emph{Mobile Computing} and helped us increase our knowledge of machine learning in applications for mobile devices.

The goal we set ourselves for this project was to develop basic neural networks for classifying human activities.
We wanted to explore with how little complexity we could achieve a good result and how much impact optimization techniques had on the result.
We also wanted to experiment with more complex neural networks and see how much better they performed compared to the simpler ones, to see if the complexity is needed.
We did not want to focus on developing a model that could be used for real-time classification, but rather focus on the classification itself.

The following sections detail our approach to the project such as related work, the methodologies we employed, the results we obtained, and the conclusions drawn from our findings.

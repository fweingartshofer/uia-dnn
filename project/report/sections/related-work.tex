\section{Related Work}\label{sec:related-work}

\gls{har} is an expanding area of research within mobile computing, which is attributed mainly to the rising availability of sensors embedded in various devices such as smartphones and wearables~\autocite{li2021deep, zhang2022sensors}.
This expansion led to growing research on this topic, providing us with many papers to use as reference.
Research on this topic shows, that human activity can be recognized by applying relatively simple machine learning methods like \gls{knn} and \gls{svm} with considerable success~\cite{Bustoni2020svmknn, zhao2018deep}.
However, such machine learning methods for \gls{har} require the selection of features by human intervention and might not even achieve sufficient performance~\cite{li2021deep}.
Recent years witnessed a growing shift in \gls{har} research towards more powerful deep learning models as a remedy for these challenges~\cite{gu2021survey}.

\gls{cnn} are renowned for their excellent capability to handle high-dimensional data and robust feature extraction ability, and have become popular for \gls{har} tasks.
They possess the intrinsic ability to conduct automatic feature extraction and selection from raw data, which greatly benefits the handling of complex sensor data.
Their robustness to local translations and distortions has made them a good fit for recognizing activities from sensor data with spatial hierarchies or where the spatial positions of patterns play a crucial role.
~\cite{zhao2018deep, yang2015deep}

Newer research documented great success with more complex and sequence-sensitive models like the \gls{lstm}.
\gls{lstm} are a type of Recurrent Neural Network (RNN) that are naturally suited to handle time-series data, making them ideal for \gls{har}, given the temporal dependencies involved.
Especially the works of~\citeauthor{zhao2018deep} have shown promise in achieving high recognition results with \gls{lstm} for \gls{har} tasks~\cite{zhao2018deep,gu2021survey}.

Deeper architectures, like \gls{dcnn}, have also found relevance in \gls{har}, demonstrating their capability to handle complex recognition tasks.
\authorcite{creagh2021dcnn}'s use of a \gls{dcnn} for distinguishing healthy individuals from participants with MS showcases an innovative use of deep learning in health-related \gls{har} applications.
This highlighted the potential of \gls{dcnn} to deliver superior results over traditional \gls{svm} methods in a challenging \gls{har} context.
~\cite{creagh2021dcnn}

Another line of research has been the use of hybrid deep models for \gls{har}.
One such instance is the EMI-RNN model, which merges the strengths of \gls{cnn} and \gls{lstm}.
This combination model expands the potential of deep learning in HAR by utilizing the depth of \gls{cnn} for feature extraction and the sequential understanding of \gls{lstm} for temporal interpretation.
~\cite{dennis2018emirnn}

Furthermore, several studies have sought to examine the comparative strengths and usage of different deep learning models for \gls{har}~\cite{watanabeGitHub,pailaGithub,bradwayGitHub}.
These studies serve as practical guides for understanding the application of these models in specific \gls{har} contexts and would eventually become a basis for our own work.
As our work focused on our personal understanding of the topic at hand, we aimed to maintain a focus on simplicity in our own model design.

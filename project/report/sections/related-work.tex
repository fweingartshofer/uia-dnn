\section{Related Work}\label{sec:related-work}

Explain \gls{cnn} and \gls{lstm}.

Intro to deep learning for \gls{har}.~\cite{Brownlee2019mlmastery}

There exist many papers that address \gls{har} problems using deep learning.

Models like \gls{knn} and \gls{svm} are already good at solving \gls{har} problems.
Papers claim NNs better distinguish between activities and are more robust to noise.
\gls{dcnn}, \gls{lstm}, and \gls{cnn} are the most popular NNs used for \gls{har} problems.

Paper achieved better results with \gls{lstm}~\cite{zhao2018deep}.

comprehensive and in-depth survey on \gls{har} with recently developed deep learning methods~\cite{gu2021survey}.

Many papers showcase the effectiveness of specific NNs in \gls{har} problems.
Some include comparisons or instructions on how to use them.
Existing papers and results from GitHub show similar effectiveness of models that we found in our study.
~\cite{watanabeGitHub,pailaGithub,bradwayGitHub}
-\gls{har}

More models we did not try:

\gls{dcnn} applied to smartphone inertial sensor data were shown to better distinguish healthy from MS participant ambulation, compared to standard \gls{svm} feature-based methodologies
~\cite{creagh2021dcnn}.

We also came across EMI-RNN, which is a model that uses a combination of \gls{cnn} and \gls{lstm} to achieve high accuracy in \gls{har} problems.~\cite{dennis2018emirnn}
\section{Conclusion}\label{sec:conclusion}

In this project we have implemented a few simple but effective models for classifying the human activities of the UCI HAR dataset.
We have shown that even the simplest models can achieve a high accuracy on this dataset and with even a little bit of tuning, we can achieve a very high accuracy.

We learned that tuning hyperparameters is crucial even for simple models, and can have a big impact on the performance.
There are a lot of different ways to tune hyperparameters, and we have only used \texttt{RandomizedSearch} in this project, but it would be interesting to try out other methods such as \texttt{GridSearch} or \texttt{BayesianSearch}.

We also learned that even with a simple smartphone, we already have more than enough data to pretty accurately classify human activities.
Which could be used, for example, in a fitness app to track the activities of the user.
It also demonstrates to us that smartphone manufacturers or app developers could easily track our activities if they wanted to.
Which is a bit scary and shows that people should be careful with what apps they install on phones and what permissions they are given.
Not every app should have access to the sensors of the phone.

The authors of the dataset recorded the data with the smartphone placed on the waist of the user instead of in their pocket.
It would be interesting to see how changing the position of the smartphone would affect the accuracy of the models.
We assume that it would require a completely retrained model, as the position of the smartphone would affect the data collected by the sensors.

We could have invested more time in getting the long-short-term-memory network and the transformer network to work.
Working with the raw data instead of the preprocessed data, which would allow for real-time classification of the activities.
With the preprocessed data we used for the other models, we need to wait for the data to be collected before we could classify the activities.
Which depending on the use case, could be a problem.

